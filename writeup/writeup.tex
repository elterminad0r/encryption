% Paper size, font size, equations align to the left
\documentclass[fleqn,a4paper,11pt]{article}
\title{Encryption assignment}
\author{Izaak van Dongen}

% Load all the configuration from a package, to keep content separate from
% markup.
\usepackage{mysty}

% The actual document content. Nowadays I would probably put this in a separate
% .tex file and \input that.
\begin{document}
    \maketitle
    \tableofcontents

    \section{The Vigen\`ere cipher (UEncrypt.pas)}
    This section will concern the unit UEncrypt.pas, which contains my utility
    functions for Vigen\`ere encryption. The unit's interface looks like this:

\begin{lstlisting}[language=Pascal, caption=UEncrypt interface]
unit UEncrypt;

interface

uses SysUtils;

const
    alph_s: string = 'ABCDEFGHIJKLMNOPQRSTUVWXYZ';
    alpha: set of char = ['A'..'Z', 'a'..'z'];
    upper: set of char = ['A'..'Z'];

function vigenere(pt: ansistring; pass: string): ansistring;
function un_vigenere(pt: ansistring; pass: string): ansistring;
function proper_mod(a, b: integer): integer;
\end{lstlisting}

    The Vigen\`ere cipher is a generalisation of the Caesar cipher. Because of
    this, in fact if I implement a Vigen\`ere routine, I will also have
    implement the Caesar shift algorithm. It would also implement the one-time
    pad, as this is simply a Vigen\`ere cipher with a plaintext-length key. I
    will write my code to work on strings, as strings are guaranteed to be more
    easily `seekable'. This is important as for later programs, I'm planning to
    do a lot of backwards-and-forwards reading through the ciphertext. Ideally
    I would have written the routine to work with something like `an iterable
    of characters', but I have to make some compromises in Pascal. Also, I will
    absolutely not perform any kind of special operation for `lines' - CR;LF
    characters are simply treated as punctuation so remain intact. This is
    crucial as the `phase' of the cipher is persistent between lines.

    A short-lived byproduct of this was that any input files were mysteriously
    cut off. Inspection with wc showed that each file was cut off to exactly
    255 characters. Apparently, in some fpc modes `string' is aliased to
    `shortstring', which can only store so many characters. Because of this, I
    explicitly use `ansistring' to represent plaintexts and ciphertexts, for
    clarity, rather than having to use specific compiler options.

    Because the Vigen\`ere algorithm is so fundamentally modular, I also needed
    to write a proper modulo function, such as in Python (that guarantees a
    positive result for any input). It looks like this:

\begin{lstlisting}[language=Pascal, caption=Proper modulo function]
function proper_mod(a, b: integer): integer;
begin
    proper_mod := a mod b;
    if proper_mod < 0 then
        proper_mod := proper_mod + b;
end;
\end{lstlisting}

    Because this function complies with the strict (useful) definition of
    modulo, this means the later encryption routines that rely on it can be
    used with any combination of characters, not just words, without worrying
    about going out of bounds. Note that almost all of them `rely' on it as the
    following function will use it.

    I also wrote a function to retrieve the `alphabetic ordinal' of a letter
    (ie an integer \(k: 0 \leq k < 26\)):

\begin{lstlisting}[language=Pascal, caption=Proper modulo function]
function alpha_ord(c: char): integer;
begin
    alpha_ord := proper_mod(ord(upcase(c)) - 65, 26);
end;
\end{lstlisting}

    Note that the previous two functions both behave in a `0-indexed' kind of
    way - this is because the Vigen\`ere cipher relies heavily on modulo and
    0-indexing (A is considered to have the value 0, for example, and the
    wrapping is implemented by modulo). Because of this, when I'm working with
    Pascal's actual (1-indexed) strings, careful consideration is needed to
    avoid off-by-one errors.

    Anyway, here's the moment you've all been waiting for - the Vigen\`ere
    routine:

\begin{lstlisting}[language=Pascal, label={lst:overflowbug}, caption=Vigen\`ere algorithm]
function vigenere(pt: ansistring; pass: string): ansistring;
var
    i: integer = 0;
    c: char;
begin
    vigenere := '';
    for c in pt do
        if c in alpha then begin
            if c in upper then
                vigenere := vigenere + chr(65 + (alpha_ord(c)
                                               + alpha_ord(pass[i + 1])) mod 26)
            else
                vigenere := vigenere + chr(97 + (alpha_ord(c)
                                               + alpha_ord(pass[i + 1])) mod 26);
            i := (i + 1) mod length(pass);
        end else
            vigenere := vigenere + c;
end;
\end{lstlisting}

    It works quite simply by iterating over the plaintext and whenever it meets
    a character, advancing the position in the key, and applying the shift
    using some modular arithmetic, using a conditional to preserve case, and
    correcting for the flaw that is 1-indexing.

    It's actually also quite important that this function operates on the
    plaintext as a holistic string - a Vigen\`ere encryption must track its
    state across lines, words, or any subset of the text other than the text
    itself. Encrypting it line by line would result in a significant amount of
    extra overhead as the function would then have to track state across calls.
    This might be done as a closure, or in this case simply by storing the
    index in a variable parameter somewhere.

    This function uses its output slot to dynamically accumulate the output,
    rather than making another variable of the same type as the return type
    named `answer' or `ct' or something. This approach will be used by several
    other functions.

    Now, having written a Vigen\`ere routine, it can in fact be recycled to
    write a decryption routine, by inverting the key, followed by delegation to
    the encryption routine. In this case, `inverting the key' means finding the
    modular additive inverse of each component.

\begin{lstlisting}[language=Pascal, caption=Vigen\`ere decryption]
function un_vigenere(pt: ansistring; pass: string): ansistring;
var
    i: integer;
begin
    for i := 1 to length(pass) do
        pass[i] := alph_s[1 + proper_mod(-alpha_ord(pass[i]), 26)];
    un_vigenere := vigenere(pt, pass);
end;
\end{lstlisting}

    \section{Reading files (UFiles.pas)}
    Having done the more fun part, we now have to briefly divert our attention
    to file handling. This unit handles all of the file reading, providing an
    interface to a single simple function:

\begin{lstlisting}[language=Pascal, caption=File-reading routine]
function read_file(var f: textfile): ansistring;
var
    c: char;
begin
    read_file := '';
    while not eof(f) do begin
        read(f, c);
        read_file := read_file + c;
    end;
end;
\end{lstlisting}

    \section{Command line interface (PVigenere.pas)}
    At last! A program that will do something. PVigenere will use the previous
    routines and interface with the command line to allow them to be used. The
    program is as follows:

\begin{lstlisting}[language=Pascal, caption=Command-line interface for Vigen\`ere routines (PVigenere.pas)]
{$MODE OBJFPC}

program PVigenere;

uses
    UEncrypt, UFiles, SysUtils;

type
    crypt_func = function(pt: ansistring; pass: string): ansistring;

var
    pass: string;
    arg_i: integer;
    encryptor: crypt_func = @vigenere;
begin
    pass := 'n';
    for arg_i := 1 to paramcount do
        if paramstr(arg_i) = '--decrypt' then
            encryptor := @un_vigenere
        else
            try
                pass := alph_s[1 + proper_mod(strtoint(paramstr(arg_i)), 26)];
            except
                on EConvertError do
                    pass := paramstr(arg_i);
            end;
    write(output, encryptor(read_file(input), pass));
end.
\end{lstlisting}
\iffalse $ \fi %syntax

    This `parses' the command-line arguments to determine what to do with what
    parameters. If it encounters the argument `-{}-decrypt' anywhere, it uses the
    Vigen\`ere decryption routine rather than the encryption routine, which it
    does by using a function variable to store the encryption routine, as both
    the encryptor and decryptor have the same signature.

    The way it determines the key is as follows: first, the key is set to the
    default value of `n'. This results in a Caesar-shift of 13, or a ROT13
    cipher. Then, for each argument that is not `-{}-decrypt', it sets to key to
    this argument, in some sense. First, it tries to parse the argument as an
    integer caesar shift. If this fails, the argument is used as a Vigen\`ere
    key. For anything other than the alphabet this behaviour may not be very
    obvious, as it uses ASCII - 65 mod 26. However, the function is still
    well-defined and if the same key is used for decryption, it will
    successfully decrypt.

    Because it must `attempt' to parse an integer, it uses a try/catch
    statement. Because of this, the mode OBJFPC must be used.

    Note also that this program is configured to work with an output file -
    although that output file is STDOUT, it still writes to a file streams
    rather than using the primitive write or writeln with no file parameter.
    Input is similarly read from STDIN. This is pretty standard for a command
    line program - if the user wants to read from actual input files, this can
    be easily done with \texttt{\$ bin/Vigenere key < input.txt > output.txt},
    or alternatively they might use the cat command. Leaving to writing to
    STDOUT is a lot easier for dynamic testing purposes, and doesn't sacrifice
    any functionality.  It should be fairly obvious that reading from STDIN
    constitutes reading from a file, but here is also a demonstration of how
    this could be used with an output file rather than writing directly to the
    terminal:

\begin{lstlisting}[caption=Use of an output file]
$ echo "There should be one-- and preferably only one --obvious way to do it." | bin/Vigenere Guido > zen.txt
$ cat zen.txt
Zbmus ybwxzj vm rbk-- uvg dxynhfgvtb ctfg rbk --ijywuoa zoe nw gc on.
$ bin/Vigenere --decrypt Guido < zen.txt
There should be one-- and preferably only one --obvious way to do it.
\end{lstlisting}
\iffalse $ \fi %syntax

    \section{Compiling and testing}

    This time I wrote another makefile, but one that works generically with my
    Pascal naming conventions. It assumes that any some \texttt{P(.*)\.pas} will
    compile to \texttt{bin/\\1}, and also has a depency on all local units, which
    take the form \texttt{U.*\.pas}.

\begin{lstlisting}[language=make, caption=The generic FPC makefile]
.PHONY: all
all: $(shell echo P*.pas | sed -E "s/P([^.]*)\.pas/bin\/\1/g")

bin/%: P%.pas $(wildcard U*.pas)
	fpc -TLinux -o$@ $<
\end{lstlisting}
\iffalse $ \fi %syntax

    Note that this in fact uses three different languages to represent a generic
    file - firstly, the generic makefile \texttt{bin/\%} and \texttt{P\%.pas},
    secondly the shell globs \texttt{U*.pas} and \texttt{P*.pas}, and lastly the
    extended regex sed command \texttt{s/P([\^.]*)\.pas/bin\/\\1/g}. The make
    generics are needed to represent the instructions to build any specific
    file. The shell glob is used to find all possible program files in a
    directory, and the regex is used to scrape the executable names from the
    program file names. Now, I can run `make', which builds the `all' target,
    which is phony so will always try to build all dependecies, which in this
    case is just `bin/Vigenere'.

    Anyway, having built my executable `bin/Vigenere', I could now verify that
    it worked correctly. The first test string I used was
    AbCdEfGhIjKlMnOpQrStUvWxYz, as this would be useful to test if shifts were
    working correctly, and if case was being preserved. Don't worry, I didn't
    type it out by hand.

\begin{lstlisting}[language=Python, caption=Alphabet]
In [2]: "".join(i.lower() if ind & 1 else i for ind, i in enumerate(string.ascii_uppercase))
Out[2]: 'AbCdEfGhIjKlMnOpQrStUvWxYz'
\end{lstlisting}

    Anyway,

\begin{lstlisting}[caption=Testing bin/Vigenere]
$ echo "AbCdEfGhIjKlMnOpQrStUvWxYz" | bin/Vigenere 1
BcDeFgHiJkLmNoPqRsTuVwXyZa
$ echo "AbCdEfGhIjKlMnOpQrStUvWxYz" | bin/Vigenere ab
AcCeEgGiIkKmMoOqQsSuUwWyYa
$ echo "AbCdEfGhIjKlMnOpQrStUvWxYz" | bin/Vigenere 3
DeFgHiJkLmNoPqRsTuVwXyZaBc
$ echo "AbCdEfGhIjKlMnOpQrStUvWxYz" | bin/Vigenere 1 | bin/Vigenere --decrypt 1
AbCdEfGhIjKlMnOpQrStUvWxYz
$ echo "AbCdEfGhIjKlMnOpQrStUvWxYz" | bin/Vigenere ab | bin/Vigenere --decrypt ab
AbCdEfGhIjKlMnOpQrStUvWxYz
\end{lstlisting}
\iffalse $ \fi %syntax

    So far, so good. To really put it to the test, I decided to try it with the
    works of Shakespeare, on the assumption that all the edge cases would
    probably be covered somewhere along the way.

\begin{lstlisting}[caption=Shakespeare]
$ cat ~/programmeren/A453/text/shakespeare.txt | wc -c
5458199
$ time cat ~/programmeren/A453/text/shakespeare.txt | bin/Vigenere "William Shakespeare" | wc -c
5458199
cat ~/programmeren/A453/text/shakespeare.txt  0.00s user 0.02s system 5% cpu 0.399 total
bin/Vigenere "William Shakespeare"  0.84s user 0.07s system 99% cpu 0.920 total
wc -c  0.00s user 0.01s system 1% cpu 0.919 total
$ time diff <(cat ~/programmeren/A453/text/shakespeare.txt |
>             bin/Vigenere "William Shakespeare" |
>             bin/Vigenere --decrypt "William Shakespeare")
>           ~/programmeren/A453/text/shakespeare.txt
diff  ~/programmeren/A453/text/shakespeare.txt  0.00s user 0.02s system 1% cpu 1.792 total
\end{lstlisting}
\iffalse $ \fi %syntax

    This actually did lead to the discovery a bug: as the works of Shakespeare
    are longer than the size of an integer, the integer (i) tracking the index
    in letters would overflow, leading to misalignment of the key and the
    plaintext. This was fixed by applying a modulus to the index whenever
    modifying it, rather than only when using it to index. See code listing
    \ref{lst:overflowbug}. This has now been fixed, and as you can see, the
    `diff' command with the source text and the decrypted plaintext exits
    cleanly, indicating it has been perfectly replicated.

    As this is simultaneously an implementation of the one-time pad, we can use
    it as such. For now it will have to be more of a proof of concept, as the
    key is limited to \(< 255\) characters, as it's stored in a string and taken
    from argv.

\begin{lstlisting}[caption=Using the Vigen\`ere program for a one-time pad]
$ echo "the quick brown fox jumps over the lazy dog" | bin/Vigenere $(cat *(.))
xuh xyvfr ferdr sre nhpww bylv gkl pncf hbj
$ echo "the quick brown fox jumps over the lazy dog" | bin/Vigenere $(cat *(.)) | bin/Vigenere --decrypt $(cat *(.))
the quick brown fox jumps over the lazy dog
$ echo "the quick brown fox jumps over the lazy dog" | bin/Vigenere $(cat /dev/urandom | head -c 100)
jih shkno ugein gyr pwsce kgiu ycu mdbl fzk
\end{lstlisting}
\iffalse $ \fi %syntax

    Here I've first used the concatenation of all the files in the directory as
    a recoverable key, to demonstrate also the decryption, followed by 100
    cryptographically secure random bytes from /dev/urandom resulting in an
    unbreakable (undecryptable) encryption.

    I also performed some further systematic tests after determining that the
    program could run successfully:

    % Make the table columns be in a fixed width font
    \begin{tabular}{>{\ttfamily}l >{\ttfamily}c}
        \toprule
        Command & Output \\
        \midrule
        \$ echo Dog | bin/Vigenere b & Eph \\
        \$ echo Dog | bin/Vigenere 1 & Eph \\
        \$ echo Dog | bin/Vigenere abc & Dpi \\
        \$ echo Eph | bin/Vigenere --decrypt b & Dog \\
        \$ echo xy\_z2 | bin/Vigenere 33 & ef\_g2 \\
        \$ echo ef\_g2 | bin/Vigenere --decrypt 33 & xy\_z2 \\
        \$ echo "Izaak van Dongen" | bin/Vigenere Python3.7 & Xxthy imu Tdlzlb \\
        \$ echo "Xxthy imu Tdlzlb" | bin/Vigenere --decrypt Python3.7 & Izaak van Dongen \\
        \bottomrule
    \end{tabular}

    Each of these is correct behaviour.

    \section{Determing keyword length}

    Now, having implemented the easy approach for decryption (where you know the
    keyword), I decided to write a program that, for the user's convenience,
    doesn't require them to remember their Vigen\`ere key. The first step is to
    determine the length of the keyword. This is absolutely trivial using some
    basic statistics and a tiny amount of computing power. For each hypothesised
    key length, we can simply consider each sequence of letters that would be
    encrypted by the same keyword letter - ie for a keyword of length \(l\) we
    get \(l\) different sub-texts of the ciphertext, where the \(i\)th of these
    is given by \(S_i = \{C_{nl + i}:\ n \in \mathbb{N} \ \land\ nl + i <
    \abs{c}\}\). If the hypothesis is correct, these will each have been
    individually encrypted with a single letter of the key, so  we then
    calculate the index of coincidence of the distribution of each of these
    sub-texts. The IOC is the probability of any two letters being the same.
    This has the very obvious property that if the letters change around, the
    IOC will not change, so the IOC of a text is invariant under any kind of
    monoalphabetic substitution cipher. It can be calculated from a distribution
    \(d\) indexed from 1 to 26 as follows:

\begin{align*}
    \text{IOC}\ &= \frac{\sum\limits_{i = 1}^{26} d_i (d_i - 1)}{T (T - 1)}\\
    \text{where}\ T &= \sum\limits_{i = 1}^{26} d_i\\
\end{align*}

    The IOC is a very strong indicator of natural language/English. The
    expected IOC of English is around 0.067, whereas for uniformly distributed
    text the expected distribution is \(\frac{1}{26} = 0.0385\).

    All of the cracking functions have been written in one unit, `UAttack.pas'.
    This unit has the following interface:

\begin{lstlisting}[language=Pascal, caption=UAttack interface]
unit UAttack;

interface

uses UFiles, Math;

type
    norm_dist = array['A'..'Z'] of real;
    dist = array['A'..'Z'] of integer;

const
    eng_dist: norm_dist =
    (0.08167, 0.01492, 0.02782, 0.04253, 0.12702, 0.02228, 0.02015, 0.06094,
     0.06966, 0.00153, 0.00772, 0.04025, 0.02406, 0.06749, 0.07507, 0.01929,
     0.00095, 0.05987, 0.06327, 0.09056, 0.02758, 0.00978, 0.02360, 0.00150,
     0.01974, 0.00074);
    alpha: set of char = ['A'..'Z', 'a'..'z'];

function fitness(pt_dist: norm_dist): real;
function IOC(d: dist): real;
function get_dist(pt: ansistring): dist;
function get_interval(pt: ansistring; start, interval: integer): ansistring;
function clean(pt: ansistring): ansistring;
function normalise(d: dist): norm_dist;
\end{lstlisting}

    All of these functions will be covered in due course. Things to note are
    the `dist' and `norm\_dist' types, which represent a discrete distribution
    and a normalised probability distribution respectively (ie `dist' directly
    represents letter frequencies whereas `norm\_dist' is normalised so that
    its sum is 1). `eng\_dist' is the distribution of English, taken from
    Wikipedia.

    Now, the first thing you need to do if you want to perform this analysis by
    IOC is to actually get our \(S_i\) substrings. For this, first we'll want to
    strip away all characters we aren't interested in. This is what the `clean'
    function does:

\begin{lstlisting}[language=Pascal, caption=Clean function]
function clean(pt: ansistring): ansistring;
var
    c: char;
begin
    clean := '';
    for c in pt do
        if c in alpha then
            clean := clean + c;
end;
\end{lstlisting}

    Now, we can extract the sub-text, calculate its distribution, and calculate
    its distribution's IOC. This part is where the use of strings rather than
    file streams becomes really crucial, as we want to separately examine
    different parts of the strings, repeatedly.

\begin{lstlisting}[language=Pascal, caption=The rest of the owl (IOC functions)]
function get_interval(pt: ansistring; start, interval: integer): ansistring;
var
    i: integer;
begin
    get_interval := '';
    for i := 0 to (length(pt) - start) div interval do
        get_interval := get_interval + pt[i * interval + start + 1];
end;

function get_dist(pt: ansistring): dist;
var
    c: char;
begin
    for c := 'A' to 'Z' do
        get_dist[c] := 0;
    for c in pt do
        if c in alpha then
            inc(get_dist[upcase(c)]);
end;

function IOC(d: dist): real;
var
    total: integer = 0;
    freq: integer;
begin
    IOC := 0;
    for freq in d do begin
        IOC := IOC + freq * (freq - 1);
        total := total + freq;
    end;
    IOC := IOC / ((total * (total - 1)));
end;
\end{lstlisting}

    The only thing of real note here is that the IOC function only makes one
    pass, where it accumulates its denominator and the total \(T\) value.

    Now, I needed to write a program that put these functions to use.

\begin{lstlisting}[language=Pascal, caption=Command-line interface to IOC functions (PIOC.pas)]
program PIOC;

uses UAttack, UFiles, Sysutils, Strutils;

function interval_ioc(pt: ansistring; interval: integer): real;
var
    start: integer;
begin
    interval_ioc := 0;
    for start := 0 to interval - 1 do
        interval_ioc := interval_ioc + IOC(get_dist(get_interval(pt, start, interval)));
    interval_ioc := interval_ioc / interval;
end;

procedure print_IOC(pt: ansistring; max_interval: integer);
var
    interval: integer;
begin
    for interval := 1 to max_interval do begin
        writeln(format('%2d (%.4f) %s', [interval,
                                         interval_ioc(pt, interval),
                                         dupestring('-',
                                                    trunc(interval_ioc(pt, interval) * 500))]));
    end;
end;

begin
    print_ioc(clean(read_file(input)), 20);
end.
\end{lstlisting}

    This firstly calculates the average IOC for each of the \(l\) sub-texts for a
    hypothesised key length \(l\), and then displays this as a bar graph of
    \(l\)
    against IOC. My actual source code uses the character U+2796, or `HEAVY
    MINUS SIGN', to produce a smooth bar graph. This uses the C-like `format'
    function to ensure that everything is properly aligned (see listing
    \ref{lst:iocchart}).

    \section{Determining the keyword}

    Now that we can assume we know the length of the keyword, we can start to
    attack the keyword itself. To do this, we will need a little brute force -
    with very low complexity. The number of possibilities that need to be
    checked as \(26l\), which is effectively constant, as for most ciphertexts
    \(l\) will be \(< 20\). Therefore the only real factor will be the length of
    the ciphertext, in which this algorithm should be roughly linear.

    To fully automate this, we will need to produce a fitness metric for some
    text. A very good metric for this is quadgram probability - for each
    quadgram in the text, use a lookup table of probabilities for quadgrams to
    occur in English, and use these to accumulate a score for the text. I've
    used this previously with simulated annealing to attack ciphers that take
    permutations of the alphabet as a key (eg substitution, playfair, bifid).
    However, this is not suitable for this algorithm because this isn't a
    holistic attack - we are attacking separate subsets of the text. Because of
    this, we won't actually form any adjacent quadgrams, so we can't use this
    method.

    The best criterion we have is `how similar does this distribution look to
    English?'. This can be very effectively represented by getting a
    probability distribution from the sub-text (this is what all of the
    `norm\_dist' stuff was about), and then for each letter taking the
    difference between its probability in standard english, its probability in
    the text, and squaring it, taking the sum of these squares. The squaring
    step is useful as it ensures only the magnitude of a difference is
    considered - ie negative differences are still differences. This could also
    have been done using the abs function, but squaring has the added advantage
    of being `harsher' for larger differences, and less harsh for smaller
    differences. Using this metric, we can determine each letter of the
    keyword. The following are functions from `UAttack.pas':

\begin{lstlisting}[language=Pascal, caption=Fitness library functions]
function normalise(d: dist): norm_dist;
var
    total: integer = 0;
    i: integer;
    dist_i: char;
begin
    for i in d do
        total := total + i;
    for dist_i := 'A' to 'Z' do
        normalise[dist_i] := d[dist_i] / total;
end;

function fitness(pt_dist: norm_dist): real;
var
    i: char;
begin
    fitness := 0;
    for i := 'A' to 'Z' do
        fitness := fitness + power(pt_dist[i] - eng_dist[i], 2);
end;
\end{lstlisting}

    And here is the calling program:

\begin{lstlisting}[language=Pascal, caption=Keyword-cracking program (PCrack.pas)]
{$MODE OBJFPC}

program PCrack;

uses UAttack, UEncrypt, UFiles, Sysutils;

function likely_caes(pt: ansistring): char;
var
    c, best_c: char;
    f, best_f: real;
begin
    best_f := -999;
    for c := 'A' to 'Z' do begin
        f := -fitness(normalise(get_dist(un_vigenere(pt, c))));
        if f > best_f then begin
            best_f := f;
            best_c := c;
        end;
    end;
    likely_caes := best_c;
end;

function likely_vig(pt: ansistring; keyl: integer): string;
var
    strpd: ansistring;
    start: integer;
begin
    strpd := clean(pt);
    likely_vig := '';
    for start := 0 to keyl - 1 do
        likely_vig := likely_vig + likely_caes(get_interval(strpd, start, keyl));
end;

var
    vg_key: string;
    pt: ansistring;
    keyl: integer;
begin
    if paramcount > 0 then
        try
            keyl := strtoint(paramstr(1));
        except
            on EConvertError do begin
                writeln(stderr, 'Invalid key length ', paramstr(1));
                exit;
            end;
        end
    else begin
        writeln(stderr, 'Key length required; see bin/IOC');
        exit;
    end;
    pt := read_file(input);
    vg_key := likely_vig(pt, keyl);
    writeln(output, 'key is ', vg_key);
    writeln(output, 'resulting pt: ', un_vigenere(pt, vg_key));
end.
\end{lstlisting}
\iffalse $ \fi %syntax

    This first actually defines a function `likely\_caes' to crack a caesar
    cipher, determining a single letter. This function is then called from
    `likely\_vig' on each caesar-encrypted subtext to find each letter of the
    keyword (this of course also works on a caesar cipher, where the length of
    the keyword is 1).

    This also reads the length of the keyword from command-line arguments,
    which the user should have determined using the IOC program, using various
    failsafes.

    \section{Cracking a cipher}

    Now I can put it to use. I will use challenge 4b from the cipher challenge
    as a demonstration. Cracking it takes literally two commands now, each
    running in a couple of milliseconds:

\begin{lstlisting}[caption=Cracking 4b, label={lst:iocchart}]
$ cat ~/programmeren/cipher_tools/src/samples/4b.txt | bin/IOC
 1 (0.0429) ---------------------
 2 (0.0429) ---------------------
 3 (0.0428) ---------------------
 4 (0.0428) ---------------------
 5 (0.0427) ---------------------
 6 (0.0428) ---------------------
 7 (0.0427) ---------------------
 8 (0.0427) ---------------------
 9 (0.0429) ---------------------
10 (0.0426) ---------------------
11 (0.0433) ---------------------
12 (0.0426) ---------------------
13 (0.0688) ----------------------------------
14 (0.0427) ---------------------
15 (0.0427) ---------------------
16 (0.0427) ---------------------
17 (0.0428) ---------------------
18 (0.0431) ---------------------
19 (0.0429) ---------------------
20 (0.0421) ---------------------
$ cat ~/programmeren/cipher_tools/src/samples/4b.txt | bin/Crack 13
key is ARCANAIMPERII
resulting pt: OVER THE YEARS THE HEROIC ROLE OF AGRICOLA AT WATLING STREET...
\end{lstlisting}

    \section{Source} All involved files, including this \LaTeX{} document, can
    be found at \url{https://github.com/elterminad0r/encryption}. An
    interested reader may also wish to explore
    \url{https://github.com/elterminad0r/cipher_tools}, the repository with
    all the code I produced for the cipher challenge this year. It includes
    automated crackers for the Hill cipher, autokey cipher and affine
    Vigen\`ere ciphers by brute force, a framework to allow a user to make
    substitutions to a text in aid of the cracking of a generic substitution
    cipher, and crackers for substitution ciphers, playfair ciphers and bifid
    ciphers written in C using simulated annealing (including quadgram
    scoring). It also features a slightly derelict Python script that tries to
    split punctuation-stripped text into words using a prefix tree. It has
    about 2639 non-empty lines of source code, in contrast to this project's
    251 lines.

\end{document}
